
\documentclass[]{aastex631}

%% The default is a single spaced, 10 point font, single spaced article.
%% There are 5 other style options available via an optional argument. They
%% can be invoked like this:
%%
%% \documentclass[arguments]{aastex631}
%% 
%% where the layout options are:
%%
%%  twocolumn   : two text columns, 10 point font, single spaced article.
%%                This is the most compact and represent the final published
%%                derived PDF copy of the accepted manuscript from the publisher
%%  manuscript  : one text column, 12 point font, double spaced article.
%%  preprint    : one text column, 12 point font, single spaced article.  
%%  preprint2   : two text columns, 12 point font, single spaced article.
%%  modern      : a stylish, single text column, 12 point font, article with
%% 		  wider left and right margins. This uses the Daniel
%% 		  Foreman-Mackey and David Hogg design.
%%  RNAAS       : Supresses an abstract. Originally for RNAAS manuscripts 
%%                but now that abstracts are required this is obsolete for
%%                AAS Journals. Authors might need it for other reasons. DO NOT
%%                use \begin{abstract} and \end{abstract} with this style.
%%

%% If you want to create your own macros, you can do so
%% using \newcommand. Your macros should appear before
%% the \begin{document} command.
%%
\newcommand{\vdag}{(v)^\dagger}
\newcommand\aastex{AAS\TeX}
\newcommand\latex{La\TeX}


%% You can add a light gray and diagonal water-mark to the first page 
%% with this command:
%% \watermark{text}
%% where "text", e.g. DRAFT, is the text to appear.  If the text is 
%% long you can control the water-mark size with:
%% \setwatermarkfontsize{dimension}
%% where dimension is any recognized LaTeX dimension, e.g. pt, in, etc.
%%
%%%%%%%%%%%%%%%%%%%%%%%%%%%%%%%%%%%%%%%%%%%%%%%%%%%%%%%%%%%%%%%%%%%%%%%%%%%%%%%%
%\graphicspath{{./}{figures/}}
%% This is the end of the preamble.  Indicate the beginning of the
%% manuscript itself with \begin{document}.

\begin{document}

\title{Rubin Observatory Scheduling Algorithm and Simulations}


%% If done correctly the peer review system will be able to
%% automatically put the author and affiliation information from the manuscript
%% and save the corresponding author the trouble of entering it by hand.

%\correspondingauthor{August Muench}
%\email{greg.schwarz@aas.org, gus.muench@aas.org}

\author[0000-0003-2874-6464]{Peter Yoachim}
\affiliation{Department of Astronomy, University of Washington, Seattle, WA, USA}

\author{and more}
\affiliation{affilt}


%% Note that the \and command from previous versions of AASTeX is now
%% depreciated in this version as it is no longer necessary. AASTeX 
%% automatically takes care of all commas and "and"s between authors names.

%% AASTeX 6.31 has the new \collaboration and \nocollaboration commands to
%% provide the collaboration status of a group of authors. These commands 
%% can be used either before or after the list of corresponding authors. The
%% argument for \collaboration is the collaboration identifier. Authors are
%% encouraged to surround collaboration identifiers with ()s. The 
%% \nocollaboration command takes no argument and exists to indicate that
%% the nearby authors are not part of surrounding collaborations.

%% Mark off the abstract in the ``abstract'' environment. 
\begin{abstract}

abstract

\end{abstract}

%% Keywords should appear after the \end{abstract} command. 
%% The AAS Journals now uses Unified Astronomy Thesaurus concepts:
%% https://astrothesaurus.org
%% You will be asked to selected these concepts during the submission process
%% but this old "keyword" functionality is maintained in case authors want
%% to include these concepts in their preprints.
\keywords{methods: observational, telescopes, surveys}

%% From the front matter, we move on to the body of the paper.
%% Sections are demarcated by \section and \subsection, respectively.
%% Observe the use of the LaTeX \label
%% command after the \subsection to give a symbolic KEY to the
%% subsection for cross-referencing in a \ref command.
%% You can use LaTeX's \ref and \label commands to keep track of
%% cross-references to sections, equations, tables, and figures.
%% That way, if you change the order of any elements, LaTeX will
%% automatically renumber them.
%%
%% We recommend that authors also use the natbib \citep
%% and \citet commands to identify citations.  The citations are
%% tied to the reference list via symbolic KEYs. The KEY corresponds
%% to the KEY in the \bibitem in the reference list below. 

\section{Introduction} \label{sec:intro}
Blah blah intro

Elah's paper \citet{Naghib2019}, Daniel's paper \citep{Rothchild2019}

early attempts at simulating and scheduling Rubin \citet{Delgado2014, Delgado2016SPIE}.


\section{Model Observatory}\label{sec:model_obs}



Sky brightness paper \citep{Yoachim2016}

Run through the information the model observatory generates and passes to the scheduler.

There is an additional scheduler that controls which filters are loaded at the start of each night. This is configured so that if the moon is 40\% illuminated or more the $grizy$ filters are loaded, otherwise the $ugriy$ filters are loaded.


\section{Deep Drilling Fields}

The LSST is planned to include several deep drilling fields. xxx--each DDF, maybe a cite for other studies of them. We plan to observe the Euclid Deep Field South (EDFS) as a pair of pointings. 



\begin{table}
    \centering
        \begin{tabular}{lcc}
        \toprule
        Short Name &      RA (deg) &     Dec (deg) \\
        \hline
           ELAISS1 &   9.450 & -44.000 \\
           XMM\_LSS &  35.708 &  -4.750 \\
             ECDFS &  53.125 & -28.100 \\
            COSMOS & 150.100 &   2.182 \\
            EDFS\_a &  58.900 & -49.315 \\
            EDFS\_b &  63.600 & -47.600 \\
        \hline
        \end{tabular}
    \caption{Adopted locations for Deep Drilling Fields.}
    \label{tab:ddf_loc}
\end{table}

\section{Markov Decision Process}






\section{Baseline Simulation}

Describe the basic decision tree that gets built. 


XXX--lay out what at baseline simulation looks like


\software{Astropy \citep{astropy2013, astropy2018}, 
          Healpy and HEALpix\footnote{\url{http://healpix.sourceforge.net}} \citep{healpix2005, Zonca2019},
          Numpy \citep{harris2020array},
          Matplotlib \citep{Hunter:2007},
          Scipy \citep{2020SciPy-NMeth},
          Pandas \citep{reback2022},
          Scikit-learn \citep{scikit-learn},
          OpenOrb \citep{Granvik2009}}
          
\bibliography{bibfile}{}
\bibliographystyle{aasjournal}



\end{document}

